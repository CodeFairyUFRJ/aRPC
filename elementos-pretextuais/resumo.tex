\begin{resumo}
\begin{SingleSpace}
A crescente adoção da arquitetura de microsserviços e a necessidade de comunicação entre linguagens distintas estimulou o desenvolvimento de novas soluções para a chamada de procedimento remoto (RPC). A diversidade de necessidades e propósitos resultou em uma variedade de implementações de RPC: algumas com foco na ergonomia de \emph{software}; outras na abrangência de linguagens e funcionalidades; e, finalmente, uma parcela visando a eficiência em computação de alto desempenho (HPC). Nesse sentido, é apresentado neste trabalho um \emph{framework} de RPC, de nome antena RPC (aRPC), com ênfase tanto no desempenho como na ergonomia de \emph{software}, inspirado no \emph{framework} gRPC, e que faz uso de novos serializadores e do protocolo de transporte QUIC para comunicação. 
Nas avaliações efetuadas, o aRPC obteve desempenho superior ao gRPC nos casos com grande quantidades de elementos nas estruturas de dados e quando os dados são mais heterogêneos e menos sintéticos. O \textit{framework} proposto consegue ser até 7\% mais rápido em relação ao gRPC, desde que as premissas descritas sejam respeitadas. Em situações com perda frequente de pacotes ou em redes de baixa qualidade, o aRPC possui desempenho muito superior ao gRPC, sendo até três vezes melhor no teste de vazão. Os resultados do aRPC abrem um campo de aplicação em sistemas de computação de alto desempenho e a resiliência apresentada faz com que seja uma opção interessante nos ambientes de IoT. Em termos gerais, o aRPC é competitivo quando comparado ao gRPC no contexto de HPC. Entretanto, o protocolo gRPC apresenta melhor desempenho para estruturas de dados mais simples e menos heterogêneas e para volumes de dados reduzidos. 


\end{SingleSpace}
\vspace{\onelineskip}
\textbf{Palavras-chave}: arpc; antenna remote procedure call; rpc; chamada remota de procedimento; remote procedure call; quic; hpc; high performace computing; computação de alto desempenho; colfer; serialização; sistemas distribuídos; thrift.

\end{resumo}
