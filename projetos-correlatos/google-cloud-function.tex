\section{Google Cloud Function}

O Google Cloud Functions se qualifica como uma forma simples de executar código em nuvem, fazendo escalonamento automático com alta tolerância a falhas que dispensa provisionamento e configurações de servidor e oferece cobrança somente quando o o código esta em execução.

\subsection{Propósito}

De acordo com a documentação da Google, os principais propósitos do Cloud Functions são, entregar ao desenvolvedor uma plataforma leve para criar funções independentes que respondam a eventos da nuvem sem necessidade de gerenciar servidores e ambientes de execução.

Os principais tipos de aplicação pensados para funcionar com essa estrutura são, \textit{Back-end} de aplicações sem servidor como sites e aplicativos, processamento de dados, arquivos, \textit{streams} e processos de ETL, além de aplicativos inteligentes, como assistentes virtuais, \textit{chatbots}, analise de video, imagens e analise de sentimento.

\subsection{Implantação}
Para a implantação de um novo código para automação, processamento ou outro proposito, deve-se pensar não a nível de sistema, mas de funções isoladas com comportamentos finalidade única e bem estabelecida. 

\bigskip
Cada uma dessas funções ou comportamentos deve ser escrita na linguagem escolhida dentre uma curta lista de opções. 

\bigskip
De acordo com o passo-a-passo da Google, para implantar uma função no cloud functions deve-se seguir 5 passos:
\begin{alineas}
	\item Preparação
	\item Criar uma cloud function
	\item Escrever o código da função
	\item Testar a função
	\item Escrever o código da função
	\item Observar registros
\end{alineas}



\subsection{Escalabilidade}

TODO

\subsection{Flexibilidade}

TODO

\subsubsection{Linguagens}

TODO

\subsubsection{Entrada e saída}

TODO

\subsection{Arquitetura}

TODO

\subsubsection{Protocolos de rede}

TODO

\subsubsection{Tolerância à falhas}

TODO

\subsection{Licença}

TODO

\subsection{Preço}

TODO

\subsection{Comparação}

TODO
